\chapter{Summary and Conclusion}\label{chap:conclusion}

In this thesis we have presented a novel methodology for data
mining in indefinite and temporal databases. We have demonstrated
throughout this thesis how NDs are useful within the data mining
process. In the thesis we have provided empirical
evidence that our dynamic use of resampling is effective for
determination of a sample size; this may have applications for other
NP-complete problems. Also, we have shown that our temporal logic for
time sequences (of NDs) is easily applicable and a viable addition to
the data mining toolkit. 



\section{Contribution of this work}\label{sec:conc_contrib}

We have outlined a general framework for data mining in non-standard
databases, not previously considered. In the most informal sense, we
take sets of approximations to FDs, in this thesis we consider only
NDs, and use statistical functions and tools to infer conclusions on
patterns within the data; in the domain of indefinite information we
use resampling in a dynamic fashion based upon mean, variance
or standard errors satisfied by sets of NDs. In the temporal
domain we can use resampling or moving averages to form new sequences
from the original values of ND set satisfaction, which change over
time, to determine specific properties which may hold within the
temporal relation sequences. Our use, in a general sense, of
statistical functions upon large sets or sequences of NDs to discover
information can be viewed as a {\em second order data
mining}. We evidence this general approach in two domains though we
speculate that there may be many more applications in other domains,
ranging from spatial to active databases.

\medskip

We now describe in more detail the specific contributions made by this work.
Chapter~\ref{chap:numdep} has shown that NDs are viable dependencies
within the relational model, extending the intentions of Grant and Minker
\cite{gm85b,gm85a} when they introduced NDs as extensions of FDs for
greater flexibility in schema specification. Principally we use the
chase for NDs, proven to be sound and complete herein, for the
inference of NDs; we show this to be decidable. In this way the chase allows for ND inference
to be tested in database applications, although this may be
intractable. We also show how NDs themselves 
may be used within data mining or database design algorithms to
approximate FD sets, demonstrated via an evolutionary database design
algorithm. NDs were shown to effectively extend the class of methods
approximating FD sets in a relation.  
ND mining may be limited in the sense that for an ND $X \to^k A$, if A
is a category of exactly $k$ elements, then the ND only
tells us that all elements occur in A; it may be considered more
informative if this were not the case, perhaps in continuous
domains. Also, the mean ND combats 
this problem by providing more information within a data mining context.
A metric for ND sets is
also provided which we employed within our work on indefinite
information in relations.

\medskip

We studied indefinite information in relations, concentrating solely
on the consistency problem, known to be NP-complete. We created a
general randomised procedure which made use of a chase developed using
NDs for indefinite relations and a dynamic resampling technique. We
chose to employ resampling to be able to make statistically valid
inferences from a sample of possible worlds taken from an indefinite
relation. Each possible world satisfies an ND set. Resampling from a
sample of possible worlds allows us to determine approximate values of
variance and standard deviation. Our randomised algorithms require a
sufficient sample size upon which to apply their selection functions
so as to obtain decent approximations to FD set satisfaction. We found
that as the variance and standard deviation change with the degree of
indefinite cells in a relation it is possible to apply resampling
iteratively on increasing sample sizes until an approximate fixpoint
is reached. Independent of our work, \cite{jl96} argue, in a position
paper, for dynamic sampling to be adopted within data mining instead of
naive sampling techniques in use. Our work does just this. Extensive
simulations on these methods showed that the chase is of use in
a larger relations with correspondingly larger domain sizes and that
the resampling is useful for providing an 
upper bound on the number of possible worlds required.

\medskip

In Chapters~\ref{chap:templog} and~\ref{chap:tempresult} we
demonstrate the practicality of NDs in temporal databases. Given that
changing ND sets, from a user supplied template, may only vary on
their branching factor we can view the sequence of changes as a time
series. Considering specific time series analysis techniques as a
basis we developed a logic using modal operators to discover rules
which a sequence may satisfy. Necessary restriction of the formulae of
our temporal logic to properties, used within program verification,
proved to be highly useful for knowledge discovery. We make no grand
claims on the formalisation of our logic with respect to it being a
panacea for time series data mining though we note that it allows for
knowledge to be represented succinctly and has an easily
understandable semantics; an important yet understated factor of many
knowledge discovery systems. Further theoretical analysis of the
logic, outside the 
scope of this thesis, is definitely required. It is most likely that
logics for time series analysis could be developed in many different
ways.

\medskip

Properties of temporal logic which were defined for program
verification have been extended for data mining purposes. The
specifications required in programs for correctness analysis lends
itself well to knowledge discovery where changing inputs over time may
frequently satisfy similar conditions. Properties of temporal logic
have not, in the limits of our experience, been considered for data
mining. 

\medskip

As we have shown this thesis is a contribution to the arena of data
mining in both techniques and tools. We show that NDs are valuable within
data mining and believe that the techniques of our randomised algorithms,
dynamic resampling, and temporal logic have clear application.  We feel
that our hypothesis of NDs for data mining in non-standard relations
has been vindicated via the work demonstrated herein.

\section{Applications}

There are a number of applications within which this work can be
used, which we now detail:
\begin{itemize}
\item Our general framework can be transferred to other domains. For
example, in a spatial database we can, after input of a FD set as a
template, mine for ND set satisfaction of this template and
then employ (or develop) statistics which are pertinent to spatial
data sets; \cite{kah96} presents $k$-predicates for spatial data
representation of the form, for example, $close\_to(x,lake) \wedge
close\_to(x,road)$ implying that $x$ is {\em close to} both a lake and a
road which could also be summarised as an ND object $\to^k$ site in a
relation CLOSE\_TO(object,site).  We believe that we could discover
and use patterns represented
by such NDs in spatial databases.
\item We can employ dynamic resampling to generate a {\em
representative} sample size in a number of NP-complete problems. 
\item Our logic can be applied to any time series for property
detection. 
\item We can mine any database for ND set satisfaction. The metric
presented in Chapter~\ref{chap:numdep} can be applied to any set of
NDs, assuming a finite domain.
\end{itemize}

\section{Directions for future research}

There are many directions for possible future research posed by this
work, in domains of dependency theory (for data mining), temporal/time
series data mining, and indefinite data mining. We begin by
considering a direct extrapolation of this research.
 
\subsection{Open Problems}

This thesis has the following {\em important} open problems:
\begin{itemize}
\item The implementation of efficient mining procedures for NDs in
standard relations. Extensions for NDs to the dynamic dependencies presented
in \cite{Via87,Via88} as outlined in Section~\ref{subsec:temdat}
warrant further analysis, with regard to both database theory and data
mining research
\item A study of algorithms to create weak Armstrong Relations, as
defined in Section~\ref{subsec:nd_ar}, for
Database design purposes. 
\item A theoretical analysis of our dynamic resampling
algorithm, WORLD\_LIMIT, is required.
\item We conjecture that
implication for ND sets with the chase is an NP-complete problem. It would be
interesting to search for special classes of NDs or relations,
possibly incomplete, within
which the chase procedure is polynomial in execution time. This work
would be similar in spirit to that of \cite{ll97c}.
\item Implementation of a query system based on our temporal logic,
NDLTL.
\item An in-depth study of expressiveness of non-standard logics, such
as NDLTL, is required. This would be particularly useful with a view to
data mining applications.
\end{itemize}

We elaborate on some of these issues in the next section.

\subsection{Further work}

In the arena of NDs we could further extend their applicability by the
creation of scaling and translation functions, as used for
transformation functions in time series similarity
\cite{alss95}. These functions have direct application when we are
dealing with relations that are of vastly different sizes. As noted in
Section~\ref{sec:nd_disc} we could also investigate more sophisticated
algorithms for the mining of NDs in standard relations. This work
could make use of many heuristics including hypergraph
transversals. One example using ND semantics may be that we do not 
have to consider mining the remainder of a relation if we have found a
partition for an ND whose branching factor is greater than over half
the number of tuples in the relation. Dynamic Dependencies introduced
for FDs by \cite{Via87} would have a highly useful semantics if
extended to NDs, as motivated by the example in
Section~\ref{subsec:temdat}. It would be of value to mine corporate,
and other, databases for the presence of these relationships whereby
the branching factor of an ND may determine subsequent branching
factors of itself and other NDs later in the timeline.
	
\medskip
The work on searching for a satisfying possible world within an
indefinite relation provides numerous avenues for further
study. Clearly, it would be highly interesting to use real-world
scheduling representations to see how useful our ND approximation sets
are. We could also 
analyse rates of convergence for our resampling process with respect to the
nature of an indefinite relation and the FD set
used. Further study of this within
such dynamic algorithms as our procedure would be very useful, both in
terms of data mining and of relevance to a multi-disciplinary research
field. Additionally, phase transitions in indefinite
relations, referred to in Section~\ref{sec:cp_disc}, would be a most
interesting further study, complementing previous 
phase transition work with dependencies and relations that have a real
information content. 

\medskip

Finally, our work on temporal data mining requires a thorough study of
the logic we have created. The inclusion of time series functionality
makes the expressive nature of the logic unclear. The flexibility of
the logic means that it is easily extended. Further
research into time series behaviour may provoke the need for
additional operators. We believe that this would include functions
designed specifically for the analysis of non-linear relationships.
We would also like to
be able to spend time developing sophisticated algorithms which use
this logic for temporal data mining. One such example would be to
discover a suitable sequence size upon which to conduct the data
mining process. Error functions from regression analysis could also be
incorporated into the logic.  Such would be desirable from a 
systems point of view.  

\section{The Evolution of Data Mining}

Data Mining is a rapidly expanding field, not least due to a
concentrated global effort into the extraction of information from
data. The state of the art applications are still led by recent
theoretical developments. There will be a significant increase in the
use of statistical developments within data mining products. Our use
of resampling in both the temporal and indefinite domains shows how
such novel processes can be applied easily and effectively. More data
mining tools will incorporate sampling and resampling in the quest for
information which may {\em characterise} a data set.

\medskip

There have been recent criticisms that data mining, as yet, is not
fully integrated with the database interface \cite{man97,joh97,cha98}. It is
only a matter of time before the next relational database upgrade
includes a data mining toolkit. For clarity and ease of use, there is
potential for the inclusion of such items as NDs and temporal logic. This,
and other, logics would make use of statistical functions within the 
database query language.  

\medskip
The process of data mining will mesh with databases so that predictors
and forecasting can be assessed at any time, which may be NDs or other
dependencies. These predictors themselves may be mined and
the technique of building our logic
upon dependencies as atoms is perhaps a first step in this direction. 

\section{Conclusions}

The field of knowledge discovery is rapidly expanding due to the
ever-increasing amounts of data being stored. The user-centric
processes of data mining are extending the fields of statistics,
artificial intelligence and machine learning into a new science
\cite{fu96}. Our work has made significant use of database,
statistical, and logical theory to develop a new general framework for
data mining in temporal and indefinite relations.




