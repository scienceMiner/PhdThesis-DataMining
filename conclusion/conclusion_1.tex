\chapter{Summary and Conclusion}\label{chap:conclusion}

In this thesis we have presented a number of new techniques for data
mining in indefinite and temporal databases. We have demonstrated
throughout this thesis how NDs are useful within the data mining
process. In other areas of the thesis we have provided empirical
evidence that our dynamic use of resampling and our temporal logic for
time sequences are both easily applicable and useful. 

\section{Contribution of this work}\label{sec:conc_contrib}

Chapter~\ref{chap:numdep} has shown that NDs are viable dependencies
within the relational model, extending the intentions of Grant and Minker
\cite{gm85b,gm85a} when they introduced NDs as extensions of FDs for
greater flexibility in schema specification. Principally we use the
Chase for NDs, proven to be sound and complete herein, for the
inference of NDs given that a finite axiomatisation has been shown not
to exist \cite{gm85b}. In this way the Chase allows for ND inference
to be tested in database applications. We also show how NDs themselves
may be used within data mining or data analysis algorithms to
approximate FD sets, demonstrated via an evolutionary database design
algorithm. NDs were shown to efficiently extend the class of methods for the
knowledge discovery of approximating FD sets in a relation. ND mining
is limited in the case when all the domain elements on the right hand
side of an FD are used, when A $\to^k$ B holds and $\mid DOM(B) \mid$ =
$k$. We introduced the mean ND to combat this problem which provides
additional information within data mining. A metric for ND sets is
also provided which we employed within our work on indefinite
information in relations.

\medskip

We studied indefinite information in relations, concentrating solely
on the consistency problem, known to be NP-complete. We created a
general randomised procedure which made use of a chase developed using
NDs for indefinite relations and a dynamic resampling technique. We
chose to employ resampling to be able to make statistically valid
inference from a sample of possible worlds taken from an indefinite
relation. Each possible world satisfies an ND set. Resampling from a
sample of possible worlds allows us to determine approximate values of
variance and standard deviation. Our randomised algorithms require a
sufficient sample size upon which to apply their selection functions
so as to obtain decent approximations to FD set satisfaction. We found
that as the variance and standard deviation change with the degree of
indefinite cells in a relation it is possible to apply resampling
iteratively on increasing sample sizes until an approximate fixpoint
is reached. Independent of our work, \cite{jl96} argue, in a position
paper, for dynamic sampling to be adopted for data mining instead of
naive sampling techniques in use. Our work does just this. Extensive
simulations on these methods showed that the Chase is of use in
a limited capacity and that the resampling is useful for providing an
upper bound on the number of possible worlds required.

\medskip

In Chapters~\ref{chap:templog} and~\ref{chap:tempresult} we
demonstrate the practicality of NDs in temporal databases. Given that
changing ND sets, from a user supplied template, may only vary on
their branching factor we can view the sequence of changes as a time
series. Considering specific time series analysis techniques as a
basis we developed a logic using modal operators to discover rules
which a sequence may satisfy. Necessary restriction of the formulae or
our temporal logic to properties, used within program verification,
proved to be highly useful for knowledge discovery. We make no grand
claims on the formalisation of our logic with respect to it being a
panacea for time series data mining though we note that it allows for
knowledge to be represented succinctly and is highly
practical. Further theoretical analysis of the logic, outside the
scope of this thesis, is definitely required. It is most likely that
logics for time series analysis could be developed in many different
ways.

\medskip

Properties of temporal logic which were defined for program
verification have been extended for data mining purposes. The
specifications required in programs for correctness analysis lends
itself well to knowledge discovery where changing inputs over time may
frequently satisfy similar conditions. Properties of temporal logic
have not, in the limits of our experience, been considered for data
mining. 

\medskip

As we have shown this thesis is a contribution to the arena of data
mining in both techniques and tools. We show that NDs are viable tools
for data mining and believe that the techniques of our randomised algorithms,
dynamic resampling and temporal logic have clear application.  We feel
that our hypothesis of NDs for data mining in non-standard relations
has been vindicated via the work demonstrated in this thesis.

\section{Applications}

Further Resampling Applications in Data Mining

\section{Directions for future research}

There are many directions for possible future research posed by this
work, in domains of dependency theory (for data mining), temporal/time
series data mining, and indefinite data mining. We begin by
considering a direct extrapolation of this research.
 
\subsection{Open Problems}

In the arena of NDs we could further extend their applicability by the
creation of scaling and translation functions, as used for
transformation functions in time series similarity
\cite{alss95}. These functions ahave direct application when we are
dealing with relations that are of vastly different sizes. As noted in
section~\ref{sec:nd_disc} we could also investigate more sophisticated
algorithms for the mining of NDs in standard relations. This work
could make use of many heuristics. One example may be that we do not
have to consider mining the remainder of a relation if we have found a
partition for an ND whose branching factor is greater than over half
the number of tuples in the relation.
	
\medskip
The work on searching for a satisfying possible world within an
indefinite relation provides numerous avenues for further
study. Clearly, it would be highly interesting to convert some
scheduling representations for a real-world study. We could also
analyse rates of convergence for our resampling process with respect to the
nature of an indefinite relation and the FD set
used. \cite{efr82,pw93} have shown that the Bootstrap converges with
respect to an increasing sample size; further study of this within
such dynamic settings as our application would be very useful, both in
terms of data mining and of relevance to a multi-disciplinary research
field. Phase transitions in indefinite
relations, referred to in section~\ref{sec:cp_disc}, would be a most
interesting further study, complementing previous 
phase transition work with dependencies and relations providing a real
information content. 

\medskip

Finally, our work on temporal data mining requires a thorough study of
the logic we have created. The inclusion of time series functionality
makes the expressive nature of the logic unclear. The flexibility of
the logic means that it is both easily extended. Further
research into time series behaviour may provoke the need for
additional operators. We believe that this would include functions
designed specifically for the analysis of non-linear relationships.
We would also like to
be able to spend time developing sophisticated algorithms which use
this logic for temporal data mining. One such example would be to
discover a suitable sequence size upon which to conduct the data
mining process. Error functions for regression analysis could also be
incorporated into the logic. This could be achieved with an integration with
regression analysis. This would be desirable from a 
systems point of view.  

\section{The Evolution of Data Mining}

Data Mining is a rapidly expanding field, not least due to a
concentrated global effort into the extraction of information from
data. The state of the art applications are still led by recent
theoretical developments. There will be a significant increase in the
use of statistical developments within data mining products. Our use
of resampling in both the temporal and indefinite domains shows how
such novel processes can be applied easily and effectively. More data
mining tools will incorporate sampling and resampling in the quest for
information which may {\em characterise} a data set.

\medskip

There have been recent criticisms that data mining, as yet, is not
fully integrated with the database interface \cite{man97,joh97}. It is
only a matter of time before the next relational database upgrade
includes a data mining toolkit. For clarity and ease of use, there is
potential for the inclusion of such items as NDs and the temporal logic
presented here, for clarity and ease of use for the data miner. This,
and other, logics would make use of statistical functions within the
database query language. 

\medskip
The process of data mining will mesh with databases so that predictors
and forecasting can be assessed at any time, which may be NDs or other
dependencies. These predictors
themselves may be mined, for a 2nd order data mining.
The technique of building our logic
upon the atoms of dependencies is perhaps a first step in this direction. 

\section{Conclusions}

The field of knowledge discovery is rapidly expanding due to the
ever-increasing amounts of data being stored. The user-centric
processes of data mining are extending the fields of statistics,
artificial intelligence and machine learning into a new science
\cite{fu96}. 




