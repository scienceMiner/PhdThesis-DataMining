\documentclass[12pt]{article}
\usepackage[dvips]{graphics}
\usepackage{latexsym,ucl_a4}


\setlength{\parindent}{0em}

\begin{document}
\pagestyle{empty}

\begin{table}[ht]
\begin{center}
\begin{tabular}{p{4em}p{27em}p{6em}} 
& \begin{center} {\bf UNIVERSITY OF LONDON} \\ {\underline{Abstract of Thesis}}
\end{center}  & \begin{flushright} See over for notes on completion 
\end{flushright} \\ 
\end{tabular}
\end{center}
\end{table}

\hfill \raisebox{-2ex}{YOUR NAME HERE} \hfill\hfill

Author (full names) \dotfill 

\hfill \raisebox{-2.3ex}{Title here } \hfill\hfill

Title of thesis \dotfill 

\hfill\hfill \raisebox{-2.3ex}{title (continued) } \hfill\hfill\hfill\hfill\hfill

\dotfill 

\hfill\hfill\hfill \raisebox{-2.3ex}{Ph.D } \hfill

\dotfill \dotfill Degree \dotfill \\

\hrulefill \\

 
Replace this garble with YOUR OWN ABSTRACT!!!
We propose that data mining, the search for useful,
non-trivial and previously unknown information within a database,
can be successfully performed with Numerical Dependencies (NDs), a
generalisation of Functional Dependencies (FDs), to model the data,
together with resampling, a computationally intensive statistical
sampling process, which allows us to make inferences from temporal
and indefinite databases.

\medskip

We use NDs to model relations containing
temporal and indefinite information. We extend the theory of NDs
by presenting measures for data mining and generalise the chase
procedure, a method for updating a relation to satisfy a constraint
set, for NDs. We motivate NDs in real-world applications by introducing
a database design tool.

\medskip

The consistency
problem, that of attempting to find a relation satisfying a set of FDs
within an indefinite relation, known to be
NP-complete, is studied in the context of using NDs for approximation.
We employ resampling, based on taking samples of definite
relations from indefinite ones, on incremental
sample sizes until an approximate fixpoint is
reached, denoting an upper bound on the required sample
size. Extensive simulations highlight that resampling
to find upper bounds in conjunction with the chase for
indefinite relations returns valid approximate solutions.

\medskip

We also study NDs in temporal sequences of relations for knowledge
discovery purposes. Each relation within a sequence is mined for a set of NDs
which evolve with updates in data. We introduce a temporal logic for the
discovery of rules and properties within these sequences, or
subsequences, which
include statistical functions within the temporal operators for
time series analysis. We also show that time series data may be analysed
using a restricted set of the logic.
We apply discovery algorithms to both
sequences and resampled sequences, allowing smoothing for trend
detection. Investigations, presented herein, show these rules to
provide interesting and practicable results.

\newpage

\underline{Notes for Candidates} \newline

\begin{tabbing}
\hspace*{1.5cm}\=\hspace*{1.5cm}\= \kill \\
1 \> Type your abstract on the \underline{other side} of this
sheet. \\
\\
2. \> Use single-space typing. {\bf Limit your abstract to one side of the
sheet.} \\
\\
3. \> Send this copy of your abstract to: \\
\\
 \> \> The Theses Assistant, \\
 \> \> University of London Library, \\
 \> \> Senate House, \\
 \> \> Malet Street, \\
 \> \> London, \\
 \> \> WC1E 7HU \\
\\
\> This abstract should be sent to the University Library at the same
time you \\
\>  submit the copies of your thesis to the Higher Degree Examinations
Office,\\ 
\>  Room 16, Senate House.\\
\\

4. \> The University Library will circulate this sheet to ASLIB
(Association of Special \\ 
\> Libraries and Information Bureaux) for publication in the
\underline{Abstracts of Theses.} 

\end{tabbing}

\hrulefill \\

\underline{For official use}\\

Subject Panel/Specialist Group \dotfill \\

BLLD \dotfill \dotfill Date of Acceptance \dotfill \\

\hrulefill \\

\end{document}




