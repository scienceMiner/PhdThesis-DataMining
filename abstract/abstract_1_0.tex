\begin{abstract}

We propose that data mining, the search for useful,
non-trivial and previously unknown information within a database,
can be successfully performed using Numerical Dependencies (NDs), a
generalisation of Functional Dependencies (FDs), and Resampling, a
computationally intensive statistical sampling process that allows us
to make inferences about temporal and indefinite databases.

\medskip

We show how NDs are applicable in relations containing
temporal and indefinite information. We extend the theory of NDs
by presenting measures for data mining and define the chase
procedure, a method for updating a relation to satisfy a constraint
set, for NDs and motivate NDs in real-world applications by introducing
an evolutionary database design tool. The Consistency
Problem, that of attempting to find a relation satisfying a set of FDs
within an indefinite relation, known to be
NP-Complete, is studied. We provide an approximation methodology
using NDs and Resampling, based on taking samples of definite
relations from indefinite ones. We employ resampling on incremental
sample sizes until an approximate fixpoint is
reached, denoting an upper bound on the required sample
size. Extensive simulations highlight that the process of resampling
to find upper bounds in conjunction with the chase for
indefinite relations returns valid approximate solutions.

\medskip

We also study NDs in temporal sequences of relations for knowledge
discovery purposes. Each relation
within a sequence satisfies a set of NDs. We introduce a logic for the
discovery of rules and properties within these sequences, or
subsequences, which
include statistical functions within the temporal operators for
correlation analysis. We apply discovery algorithms to both
sequences and resampled sequences, allowing smoothing for trend
detection. Investigations, presented herein, show these rules to
provide interesting and practicable results.
\end{abstract}

